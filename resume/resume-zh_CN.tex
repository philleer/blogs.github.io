% !TEX TS-program = xelatex
% !TEX encoding = UTF-8 Unicode
% !Mode:: "TeX:UTF-8"

\documentclass{resume}
\usepackage{zh_CN-Adobefonts_external} % Simplified Chinese Support using external fonts (./fonts/zh_CN-Adobe/)
%\usepackage{zh_CN-Adobefonts_internal} % Simplified Chinese Support using system fonts
\usepackage{linespacing_fix} % disable extra space before next section
\usepackage{cite}

\begin{document}
\pagenumbering{gobble} % suppress displaying page number

\name{李鹏辉}

% {E-mail}{mobilephone}{homepage}
\contactInfo{(+86)195-3894-2063}{phillee2016@163.com}{https://github.com/philleer}{}

\baseInfo{广东省深圳市}{算法工程师}{}

% \section{\faGraduationCap\ 教育背景}
\section{教育背景}
\datedsubsection{\textbf{哈尔滨工业大学(深圳)} \textit{工学硕士} \hspace{\stretch{0.52}}控制科学与技术(机器视觉实验室)}{2017.9 - 2020.6}
\datedsubsection{\textbf{吉林大学} \hspace{\stretch{5.3}}\textit{工学学士} \hspace{\fill}测控技术与仪器(NIR光谱分析LAB)}{2013.9 - 2017.6}

\vspace{0.5em}

\section{专业技能}
\begin{itemize}[parsep=0.5ex]
  \item \textbf{编程语言}: C/C++, Python, Shell
  \item \textbf{操作系统,项目工程构建}: Linux/macOS/SSH/Git/CMake
  \item \textbf{关键词}: VR/3D Reconstruction/Image Processing
\end{itemize}


\section{工作经历}
\datedsubsection{\textbf{深圳市铂岩科技有限公司 | RayShaper}, 六自由度姿态估计项目组-算法工程师}{2020.07-至今}
\begin{itemize}
  \item \textbf{密集多人场景人脸识别与跟踪。}
  \item 独立负责基于开源人脸库的识别和动态追踪,减少误检和漏检的出现。
  \item 独立负责基于多目标检测的人脸识别与跟踪的进一步优化。对跨镜头场景下算法的改进和完善。
\end{itemize}

\begin{itemize}
  \item \textbf{超高清8K全景视频360度拼接。}采集四源鱼眼视频,在线完成超高清8K全景视频的360拼接,并在VR头显中展示结果。
  \item 独立负责鱼眼镜头标定算法的优化。
  \item 独立负责对基于光流的全景拼接算法的GPU实现进行优化,提升了大约20\%的性能。
  \item 负责基于FFMpeg的实时拼接算法的优化。
\end{itemize}

\begin{itemize}
  \item \textbf{子弹时间自由视角合成。}基于稀疏视角的彩色图像完成同一水平面任意虚拟视角的合成,以在保证视频质量的前提下减少现场所需摄像机数量。独立完成项目初期的基础工程框架构建与调试。
  \item 负责实现基于平面拟合和纹理因子约束的单视角深度补全。
  \item 负责完成空间相机姿态的平滑插值算法。
  \item 负责完成基于多视角三维空间映射与反向映射的新视角图像融合。
  \item 负责完成对融合后彩色图像的修复和优化。
\end{itemize}

\datedsubsection{\textbf{浙江商汤科技开发有限公司 | SenseTime}, 三维视觉与增强现实组-实习生}{2019.06-2019.09}
\begin{itemize}
  \item \textbf{基于RGBD的动态实时稠密人体表面三维重建。}利用RGBD信息实现人体表面实时动态三维稠密重建,进而支持虚拟/增强现实等三维交互功能。主要负责实现单帧多视角深度图TSDF融合及优化。
  \item 负责实现与参考帧空间的空间体素融合,有效解决了动作幅度过大时的跟踪问题。
  \item 与平台研发协作实现基于栅格化的深度图填充,改善了单视角深度的稀疏性。
\end{itemize}

\section{其他项目经历}
\datedsubsection{\textbf{融合明暗信息的多视图立体重建算法研究}, 硕士毕业课题}{2018.11-2019.09}
\begin{itemize}
  \item 独立完成基于超像素分割的平面拟合,有效提升了单视角深度图的稠密性。
  \item 独立完成联合多视图立体和环境光照明暗信息的深度优化,进而提高了弱纹理区域重建完整度。
\end{itemize}
\datedsubsection{\textbf{高精度亚像素边缘检测算法研究}, 硕士实验室项目}{2018.09-2018.11}
\begin{itemize}
  \item 负责完成边缘向量函数设计及像素级边缘点位置的确定。
  \item 针对工业环境利用Steger方法完成高精度亚像素位置检测算法。
\end{itemize}

\section{个人总结}
\begin{itemize}
  \item [>] 技术博客:https://cnblogs.com/phillee
  \item [>] 本人勤奋刻苦、乐观进取,自驱力强、保持热忱,工作负责、善于沟通,认同开放、连接、融合的三维视觉技术在未来的巨大潜力。在学习和工作中不断打磨技术基础,丰富技术栈,并时刻关注相关研究,对基于3D视觉技术的工程化解决方案有浓厚兴趣。
\end{itemize}

%% Reference
%\newpage
%\bibliographystyle{IEEETran}
%\bibliography{mycite}
\end{document}
